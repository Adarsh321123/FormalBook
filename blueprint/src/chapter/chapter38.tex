\chapter{The Dinitz problem}

\begin{definition}
  \label{ch38definition1}
  Let $\vec{G} = (V, E)$ be a directed graph. A \textit{kernel} $K \subseteq V$ is a subset of the vertices such that
\begin{itemize}
    \item[(i)] $K$ is independent in $G$, and
    \item[(ii)] for every $u \notin K$ there exists a vertex $v \in K$ with an edge $u \rightarrow v$.
\end{itemize}
\end{definition}

\begin{lemma}
  \label{ch38lemma1}
  \uses{ch38definition1}
  Let $\vec{G} = (V, E)$ be a directed graph, and suppose that for each vertex $v \in V$ we have
  a color set $C(v)$ that is larger than the outdegree, $|C(v)| \geq d^+(v) + 1$.
  If every induced subgraph of $\vec{G}$ possesses a kernel, then there exists a list coloring
  of $G$ with a color from $C(v)$ for each $v$.
\end{lemma}
\begin{proof}
We proceed by induction on $|V|$. For $|V| = 1$ there is nothing to prove. Choose a color
$c \in \mathcal{C} = \bigcup_{v \in V} C(v)$ and set
\[
A(c) := \{v \in V : c \in C(v)\}.
\]
By hypothesis, the induced subgraph $G_{A(c)}$ possesses a kernel $K(c)$. Now we color all
$v \in K(c)$ with the color $c$ (this is possible since $K(c)$ is independent),
and delete $K(c)$ from $G$ and $c$ from $C$. Let $G'$ be the induced subgraph of $G$ on
$V \setminus K(c)$ with $C'(v) = C(v) \setminus \{c\}$ as the new list of color sets.
Notice that for each $v \in A(c) \setminus K(c)$, the outdegree $d^+(v)$ is decreased by at
least 1 (due to condition (ii) of a kernel). So $d^+(v) + 1 \leq |C'(v)|$ still holds in $\vec{G'}$.
The same condition also holds for the vertices outside $A(c)$, since in this case the color
sets $C(v)$ remain unchanged. The new graph $G'$ contains fewer vertices than $G$, and we are done
by induction.
\end{proof}


\begin{definition}
  \label{ch38definition2}
A matching $M$ of $G = (X \cup Y, E)$ is called \textit{stable} if the following condition holds:
Whenever $uv \in E \setminus M$, $u \in X$, $v \in Y$, then either $uy \in M$ with
$y > v$ in $N(u)$ or $xv \in M$ with $x > u$ in $N(v)$, or both.
\end{definition}


\begin{lemma}
  \label{ch38lemma2}
  \uses{ch38definition2}
  A stable matching always exists.
\end{lemma}
\begin{proof}
Consider the following algorithm. In the first stage all men $u \in X$ propose to their top choice.
If a girl receives more than one proposal she picks the one she likes best and keeps him on a
string, and if she receives just one proposal she keeps that one on a string. The remaining men
are rejected and form the reservoir $R$. In the second stage all men in $R$ propose to their next
choice. The women compare the proposals (together with the one on the string, if there is one),
pick their favorite and put him on the string. The rest is rejected and forms the new set $R$.
Now the men in $R$ propose to their next choice, and so on. A man who has proposed to his last
choice and is again rejected drops out from further consideration (as well as from the reservoir).
Clearly, after some time the reservoir $R$ is empty, and at this point the algorithm stops.

\textbf{Claim.} When the algorithm stops, then the men on the strings together with the
corresponding girls form a stable matching.

Notice first that the men on the string of a particular girl move there in increasing
preference (of the girl) since at each stage the girl compares the new proposals with the
present mate and then picks the new favorite. Hence if $uv \in E$ but $uv \notin M$, then
either $u$ never proposed to $v$ in which case he found a better mate before he even got
around to $v$, implying $uy \in M$ with $y > v$ in $N(u)$, or $u$ proposed to $v$ but was
rejected, implying $xv \in M$ with $x > u$ in $N(v)$. But this is exactly the condition of a
stable matching.
\end{proof}


\begin{theorem}
  \label{ch38theorem}
  We have $\chi_\ell(S_n) = n$ for all $n$.
\end{theorem}
\begin{proof}
  \uses{ch38lemma1, ch38lemma2}
As before we denote the vertices of $S_n$ by $(i, j)$, $1 \leq i, j \leq n$. Thus $(i, j)$ and
$(r, s)$ are adjacent if and only if $i = r$ or $j = s$. Take any Latin square $L$ with letters
from $\{1, 2, \dots, n\}$ and denote by $L(i, j)$ the entry in cell $(i, j)$. Next make $S_n$ into
a directed graph $\vec{S}_n$ by orienting the horizontal edges $(i, j) \rightarrow (i, j')$ if
$L(i, j) < L(i, j')$ and the vertical edges $(i, j) \rightarrow (i', j)$ if $L(i, j) > L(i', j)$.
Thus, horizontally we orient from the smaller to the larger element, and vertically the other
way around. (In the margin we have an example for $n = 3$.)

Notice that we obtain $d^+(i, j) = n - 1$ for all $(i, j)$. In fact, if $L(i, j) = k$, then $n - k$
cells in row $i$ contain an entry larger than $k$, and $k - 1$ cells in column $j$ have an entry
 smaller than $k$.

By Lemma \ref{ch38lemma1} it remains to show that every induced subgraph of $\vec{S}_n$ possesses a kernel.
Consider a subset $A \subseteq V$, and let $X$ be the set of rows of $L$, and $Y$ the set of
its columns. Associate to $A$ the bipartite graph $G = (X \cup Y, A)$, where every $(i, j) \in A$ is
represented by the edge $ij$ with $i \in X, j \in Y$. In the example in the margin the cells of $A$
are shaded.

The orientation on $S_n$ naturally induces a ranking on the neighborhoods in $G = (X \cup Y, A)$
by setting $j' > j$ in $N(i)$ if $(i, j) \rightarrow (i, j')$ in $\vec{S}_n$
respectively $i' > i$ in $N(j)$ if $(i, j) \rightarrow (i', j)$. By Lemma \ref{ch38lemma2},
 $G = (X \cup Y, A)$
possesses a stable matching $M$. This $M$, viewed as a subset of $A$, is our desired kernel!
To see why, note first that $M$ is independent in $A$ since for edges in $G = (X \cup Y, A)$
they do not share an endvertex $i$ or $j$. Secondly, if $(i, j) \in A \setminus M$, then by the
definition of a stable matching there either exists $(i, j') \in M$ with $j' > j$ or
$(i', j) \in M$ with $i' > i$, which for $\vec{S}_n$ means $(i, j) \rightarrow (i, j') \in M$
 or $(i, j) \rightarrow (i', j) \in M$, and the proof is complete.
\end{proof}
