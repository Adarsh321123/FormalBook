\chapter{Pigeon-hole and double counting}
\label{chapter28}
\begin{theorem}[Pigeon-hole princinple]
  \label{pigeon_hole_principle}
  \lean{chapter28.pigeon_hole_principle}
  If $n$ objects are placed in $r$ boxes, where $r < n$, then at least one of the boxes contains
  more than one object.
\end{theorem}
\begin{proof}
  \leanok
  obvious
\end{proof}

\begin{theorem}[Double counting]
  \label{double_counting}
  Suppose that we are given two finite sets $R$ and $C$ and a subset $S \subseteq R \times C$.
  Whenever $(p, q) \in S$, then we say $p$ and $q$ are incident.
  If $r_p$ denotes the number of elements that are incident to $p \in R$, and $c_q$
  denotes the number of elements that are incident to $q \in C$, then
  \[
  \sum_{p \in R} r_p = |S| = \sum_{q \in C} c_q. \tag{3}
  \]
\end{theorem}
\begin{proof}
  ``nothing to prove''
\end{proof}

\section{Numbers}
\begin{theorem}[Claim]
  \label{ch28claim1}
  Consider the numbers $1, 2, 3, \dots 2n$, and take away $n + 1$
  of them. Then there are two among these $n + 1$ numbers which are
  relatively prime.
\end{theorem}
\begin{proof}
  \uses{chapter28.pigeon_hole_principle}
  obvious
\end{proof}

\begin{theorem}[Claim]
  \label{ch28claim2}
  Suppose again $A\subset\{1, 2, \dots, 2n\}$ with $|A| = n + 1$. Then
  there are always two numbers in $A$ such that one divides the other.
\end{theorem}
\begin{proof}
  \uses{chapter28.pigeon_hole_principle}
  Write every number $a \in A$ in the form $a = 2^k m$, where $m$ is an odd number between $1$
  and $2n - 1$. Since there are $n + 1$ numbers in $A$, but only $n$ different odd parts, there must
  be two numbers in $A$ with the same odd part. Hence one is a multiple of the other.
\end{proof}

\section{Sequences}

\begin{theorem}[Claim]
  \label{ch28claim3}
  In any sequence $a_1, a_2, \ldots, a_{mn+1}$ of $mn+1$ distinct real numbers, there exists an increasing subsequence
\[
a_{i_1} < a_{i_2} < \cdots < a_{i_{m+1}} \quad (i_1 < i_2 < \cdots < i_{m+1})
\]
of length $m+1$, or a decreasing subsequence
\[
a_{j_1} > a_{j_2} > \cdots > a_{j_{n+1}} \quad (j_1 < j_2 < \cdots < j_{n+1})
\]
of length $n+1$, or both.
\end{theorem}
\begin{proof}
  \uses{chapter28.pigeon_hole_principle}
This time the application of the pigeon-hole principle is not immediate.
Associate to each $a_i$ the number $t_i$, which is the length of a longest
increasing subsequence starting at $a_i$. If $t_i \geq m+1$ for some $i$,
then we have an increasing subsequence of length $m+1$. Suppose then
that $t_i \leq m$ for all $i$. The function $f : a_i \mapsto t_i$ mapping
$\{a_1, \dots, a_{mn+1}\}$ to $\{1, \dots, m\}$ tells us by (1) that there
is some $s \in \{1, \dots, m\}$ such that $f(a_i) = s$ for $\frac{mn}{m} + 1 = n + 1$
numbers $a_i$. Let $a_{j_1}, a_{j_2}, \dots, a_{j_{n+1}}$ $(j_1 < \cdots < j_{n+1})$
be these numbers. Now look at two consecutive numbers $a_{j_i} < a_{j_{i+1}}$,
then we would obtain an increasing subsequence of length $s$ starting
at $a_{j_{i+1}}$, and consequently an increasing subsequence of length $s+1$
starting at $a_{j_i}$, which cannot be since $f(a_{j_i}) = s$.
We thus obtain a decreasing subsequence $a_{j_1} > a_{j_2} > \cdots > a_{j_{n+1}}$
of length $n+1$.
\end{proof}

\section{Sums}

\begin{theorem}[Claim]
  \label{ch28claim4}
  Suppose we are given $n$ integers $a_1, \dots, a_n$, which need not be distinct.
  Then there is always a set of consecutive numbers $a_{k+1}, a_{k+2}, \dots, a_{\ell}$
  whose sum $\sum_{i=k+1}^{\ell} a_i$ is a multiple of $n$.
\end{theorem}
\begin{proof}
  \uses{chapter28.pigeon_hole_principle}
  For the proof we set $N = \{0, 1, \dots, n\}$ and $R = \{0, 1, \dots, n-1\}$.
  Consider the map $f : N \to R$, where $f(m)$ is the remainder of $a_1 + \cdots + a_m$
  upon division by $n$. Since $|N| = n+1 > n = |R|$,
  it follows that there are two sums $a_1 + \cdots + a_k, a_1 + \cdots + a_{\ell}$ $(k < \ell)$
  with the same remainder, where the first sum may be the empty sum denoted by $0$. It follows that
  \[
  \sum_{i=k+1}^{\ell} a_i = \sum_{i=1}^{\ell} a_i - \sum_{i=1}^{k} a_i
  \]
  has remainder 0 — end of proof.
\end{proof}

\section{Numbers again}
 TODO

\section{Graphs}

\begin{lemma}[Handshaking]
  \label{handshaking}
  \lean{chapter28.handshaking}
  \leanok
  Let $G$ be a finite simple graph with vertex set $V$ and edge set $E$ and let
  $d(v)$ denote the degree of a vertex $v$. Then
  \[
  \sum_{v\in V}d(v) = 2|E|.
  \]
\end{lemma}
\begin{proof}
\leanok
For the proof consider $S \subseteq V \times E$, where $S$ is
the set of pairs $(v, e)$ such that $v \in V$ is an end-vertex
of $e \in E$. Counting $S$ in two ways gives on the one hand
$\sum_{v \in V} d(v)$, since every vertex contributes $d(v)$
to the count, and on the other hand $2|E|$, since every edge
has two ends.
\end{proof}

\begin{theorem}
  \label{ch28theorem}
  If the graph $G$ on $n$ vertices contains no $4$-cycles, then
  \[
  |E| \le \lfloor\frac{n}{4}(1 + \sqrt{4n - 3})\rfloor
  \]
\end{theorem}
\begin{proof}
  \uses{double_counting}
  TODO
\end{proof}

\section{Sperner's Lemma}
\begin{lemma}[Sperner's Lemma]
  \label{sperner}
  Suppose that some “big” triangle with vertices $V_1, V_2, V_3$ is triangulated (that is, decomposed into a finite number of “small” triangles that fit together edge-by-edge).
Assume that the vertices in the triangulation get “colors” from the set $\{1, 2, 3\}$ such that $V_i$ receives the color $i$ (for each $i$), and only the colors $i$ and $j$ are used for vertices along the edge from $V_i$ to $V_j$ (for $i \neq j$), while the interior vertices are colored arbitrarily with 1, 2, or 3.
Then in the triangulation there must be a small “tricolored” triangle, which has all three different vertex colors.
\end{lemma}
\begin{proof}
  \uses{chapter28.pigeon_hole_principle}
  We will prove a stronger statement: The number of tricolored triangles is not only nonzero,
  it is always \emph{odd}.

Consider the dual graph to the triangulation, but don’t take all its edges — only those which cross an edge that has endvertices with the (different) colors 1 and 2. Thus we get a “partial dual graph” which has degree 1 at all vertices that correspond to tricolored triangles, degree 2 for all triangles in which the two colors 1 and 2 appear, and degree 0 for triangles that do not have both colors 1 and 2. Thus only the tricolored triangles correspond to vertices of odd degree (of degree 1).

However, the vertex of the dual graph which corresponds to the outside of the triangulation has odd degree: in fact, along the big edge from $V_1$ to $V_2$, there is an odd number of changes between 1 and 2. Thus an odd number of edges of the partial dual graph crosses this big edge, while the other big edges cannot have both 1 and 2 occurring as colors.

Now since the number of odd-degree vertices in any finite graph is even (by equation (4)), we find that the number of small triangles with three different colors (corresponding to odd inside vertices of our dual graph) is odd.
\end{proof}

\begin{theorem}[Brower's Fixpoint (for $n=2$)]
  \label{brower}
  Every continuous function $f : B^2 \longrightarrow B^2$ of an $2$-dimensional ball to
  itself has a fixed point (a point $x \in B^2$ with $f(x) = x$).
\end{theorem}
\begin{proof}
  \uses{sperner}
  TODO
\end{proof}
