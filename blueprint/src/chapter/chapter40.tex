\chapter{How to guard a museum}

\begin{theorem}
  \label{museum_guards}
  For any museum with $n$ walls, $\lfloor \frac{n}{3} \rfloor$ guards suffice.
\end{theorem}
\begin{proof}
  First of all, let us draw $n - 3$ noncrossing diagonals between corners of the walls until the
  interior is triangulated. For example, we can draw 9 diagonals in the museum depicted in the
  margin to produce a triangulation. It does not matter which triangulation we choose, any one
  will do. Now think of the new figure as a plane graph with the corners as vertices and the
   walls and diagonals as edges.

\textbf{Claim.} \textit{This graph is 3-colorable.}

For $n = 3$ there is nothing to prove. Now for $n > 3$ pick any two vertices $u$ and $v$ which are
connected by a diagonal. This diagonal will split the graph into two smaller triangulated graphs
both containing the edge $uv$. By induction we may color each part with 3 colors where we may
choose color 1 for $u$ and color 2 for $v$ in each coloring. Pasting the colorings together
yields a 3-coloring of the whole graph.

The rest is easy. Since there are $n$ vertices, at least one of the color classes, say the
 vertices colored 1, contains at most $\left\lfloor \frac{n}{3} \right\rfloor$ vertices, and
  this is where we place the guards. Since every triangle contains a vertex of color 1 we infer
  that every triangle is guarded, and hence so is the whole museum.

  %TODO: show triangulation exists!
\end{proof}
