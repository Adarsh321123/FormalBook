\chapter{Bertrand's postulate}

\begin{theorem}
    \label{thm:bertrands_postulate}
    \lean{chapter2.exists_prime_lt_and_le_two_mul}
    \leanok
    For any positive natural number, there is a prime which is greater than
    it, but no more than twice as large.
\end{theorem}
\begin{proof}
    \leanok
    TODO: make this follow the book proof more closely!
\end{proof}

\section{Appendix: Some estimates}

\begin{theorem}
    \label{thm:estimate_integral}
    \lean{chapter2.harmonic_number_bounds}
    \leanok
    For all \(n \in \mathbb{N}\)
    \[
    \log n + \frac 1 n < H_n < \log n + 1.
    \]
\end{theorem}
\begin{proof}
    TODO
\end{proof}


\begin{theorem}
    \label{thm:estimate_factorials}
    \lean{chapter2.bound_factorial}
    \leanok
    For all \(n \in \mathbb{N}\)
    \[
    n! = n(n -1)! < ne^{n \log n - n + 1}= e\left(\frac n e\right)^n.
    \]
\end{theorem}
\begin{proof}
    TODO
\end{proof}

\begin{theorem}
    \label{thm:estimate_binomial_coefficient}
    \lean{chapter2.bound_binomial_coeff}
    \leanok
    \[\binom{n}{k} \le \frac{n^k}{k!} \le \frac{n^k}{2^{k - 1}}\]
\end{theorem}
\begin{proof}
    TODO
\end{proof}
