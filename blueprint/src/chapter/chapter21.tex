\chapter{The fundamental theorem of algebra}

\begin{lemma}
  \label{argand_inequality}
  Let $p(z) = \sum_{k=0}^n c_k z^k$ be a complex polynomial of degree $n\ge 1$.
  If $p(a)\ne0$, then every disk $D$ around $a$ contains an interior point $b$ with $|p(b)| < |p(a)|$
\end{lemma}
\begin{proof}
  % note: the proof here is slightly different in the sixth edition compared to the fifth! Let's
  % follow the sixth edition.
  TODO
\end{proof}

\begin{theorem}
  \label{fundamental_theorem_of_algbra}
  \lean{fundamental_theorem_of_algebra}
  Every nonconstant polynomial with complex coefficients has at least one root in the field of complex numbers.
\end{theorem}
\begin{proof}
  \uses{argand_inequality}
  The rest is easy. Clearly, $p(z)z^{-n}$ approaches the leading coefficient $c_n$
  of $p(z)$ as $|z|$ goes to infinity. Hence $|p(z)|$ goes to infinity as well with
  $|z| \to \infty$. Consequently, there exists $R_1 > 0$ such that $|p(z)| > |p(0)|$ for
  all points $z$ on the circle $\{ z : |z| = R_1 \}$. Furthermore, our third fact (C)
  tells us that in the compact set $D_1 = \{ z : |z| \leq R_1 \}$ the continuous real-valued
  function $|p(z)|$ attains the minimum value at some point $z_0$. Because
  of $|p(z)| > |p(0)|$ for $z$ on the boundary of $D_1$, $z_0$ must lie in the interior.
  But by d'Alembert's lemma \ref{argand_inequality} this minimum value $|p(z_0)|$ must
  be $0$ — and this is the whole proof.
\end{proof}
