\chapter{Representing numbers as sums of two squares}

\begin{lemma}[Lemma 1]
  \label{ch3.lemma1}
  For primes \(p = 4m + 1\) the equation \(s^2 \equiv -1 (\mod p)\) has two
  solutions \(s \in \{1, 2, \dots, p - 1\}\), for \(p = 2\) there is one such solution, while
  for primes of the form \(p = 4m + 3\) there is no solution.
\end{lemma}
\begin{proof}
  TODO
\end{proof}


\begin{lemma}[Lemma 2]
  \label{ch3.lemma2}
  No number \(n = 4m + 3\) is a sum of two squares.
\end{lemma}
\begin{proof}
  TODO
\end{proof}

\begin{proposition}[First proof]
  \label{ch3.proposition1}
  Every prime of the form \(p = 4m + 1\) is a sum of two squares,
  that is, it can be written as \(p = x^2 + y^2\) for some natural numbers \(x,y \in \mathbb{N}\).
\end{proposition}
\begin{proof}
  \uses{ch3.lemma1}
  TODO
\end{proof}

\begin{proposition}[Second proof]
  \label{ch3.proposition2}
  Every prime of the form \(p = 4m + 1\) is a sum of two squares,
  that is, it can be written as \(p = x^2 + y^2\) for some natural numbers \(x,y \in \mathbb{N}\).
\end{proposition}
\begin{proof}
  TODO (Zagier's one line proof is in mathlib by now, follow this!)
\end{proof}

\begin{proposition}[Third proof]
  \label{ch3.proposition3}
  Every prime of the form \(p = 4m + 1\) is a sum of two squares,
  that is, it can be written as \(p = x^2 + y^2\) for some natural numbers \(x,y \in \mathbb{N}\).
\end{proposition}
\begin{proof}
  TODO
\end{proof}

\begin{theorem}
  \label{sum_of_two_squares}
  A natural number \(n\) can be represented as a sum of two squares
  if and only if every prime factor of the form \(p = 4m + 3\) appears with an
  even exponent in the prime decomposition of \(n\).
\end{theorem}
\begin{proof}
  \uses{ch3.lemma2, ch3.proposition1, ch3.proposition2, ch3.proposition3}
  TODO
\end{proof}
