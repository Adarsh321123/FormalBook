\chapter{Turán's graph theorem}

\begin{theorem}[First Proof]
  \label{ch41proof1}
If a graph $G = (V, E)$ on $n$ vertices has no $p$-clique, $p \geq 2$, then
\[
|E| \leq \left(1 - \frac{1}{p - 1}\right) \frac{n^2}{2}. \tag{1}
\]
\end{theorem}
\begin{proof}
   We use induction on $n$. One easily computes that (1) is true for $n < p$.
   Let $G$ be a graph on $V = \{v_1, \dots, v_n\}$ without $p$-cliques with a maximal number of
   edges, where $n \geq p$. $G$ certainly contains $(p - 1)$-cliques, since otherwise we could
   add edges. Let $A$ be a $(p - 1)$-clique, and set $B := V \setminus A$.

$A$ contains $\binom{p - 1}{2}$ edges, and we now estimate the edge-number $e_B$ in $B$ and the
edge-number $e_{A, B}$ between $A$ and $B$. By induction,
we have $e_B \leq \frac{1}{2} \left(1 - \frac{1}{p - 1}\right) (n - p + 1)^2$.
Since $G$ has no $p$-clique, every $v_j \in B$ is adjacent to at most $p - 2$ vertices in $A$,
and we obtain $e_{A, B} \leq (p - 2)(n - p + 1)$. Altogether, this yields
\[
|E| \leq \binom{p - 1}{2} + \frac{1}{2} \left(1 - \frac{1}{p - 1}\right) (n - p + 1)^2 + (p - 2)(n - p + 1),
\]
which is precisely $\left(1 - \frac{1}{p - 1}\right) \frac{n^2}{2}$.
\end{proof}

\begin{theorem}[Second Proof]
  \label{ch41proof2}
If a graph $G = (V, E)$ on $n$ vertices has no $p$-clique, $p \geq 2$, then
\[
|E| \leq \left(1 - \frac{1}{p - 1}\right) \frac{n^2}{2}. \tag{1}
\]
\end{theorem}
\begin{proof}
  This proof makes use of the structure of the Turán graphs. Let $v_m \in V$ be a vertex of
  maximal degree $d_m = \max_{1 \leq j \leq n} d_j$. Denote by $S$ the set of neighbors of
  $v_m$, $|S| = d_m$, and set $T := V \setminus S$. As $G$ contains no $p$-clique, and $v_m$ is
  adjacent to all vertices of $S$, we note that $S$ contains no $(p - 1)$-clique.

We now construct the following graph $H$ on $V$ (see the figure). $H$ corresponds to $G$ on $S$ and
contains all edges between $S$ and $T$, but no edges within $T$. In other words, $T$ is an
independent set in $H$, and we conclude that $H$ has again no $p$-cliques. Let $d'_j$ be the
degree of $v_j$ in $H$. If $v_j \in S$, then we certainly have $d'_j \geq d_j$ by the
construction of $H$, and for $v_j \in T$, we see $d'_j = |S| = d_m \geq d_j$ by the choice of $v_m$.
We infer $|E(H)| \geq |E|$, and find that among all graphs with a maximal number of edges,
there must be one of the form of $H$. By induction, the graph induced by $S$ has at most as many
edges as a suitable graph $K_{n_1, \dots, n_{p - 2}}$ on $S$.
So $|E| \leq |E(H)| \leq E(K_{n_1, \dots, n_{p - 1}})$ with $n_{p - 1} = |T|$, which implies (1).
\end{proof}

\begin{theorem}[Third Proof]
  \label{ch41proof3}
If a graph $G = (V, E)$ on $n$ vertices has no $p$-clique, $p \geq 2$, then
\[
|E| \leq \left(1 - \frac{1}{p - 1}\right) \frac{n^2}{2}. \tag{1}
\]
\end{theorem}
\begin{proof}
Consider a \emph{probability distribution} $\mathbf{w} = (w_1, \dots, w_n)$ on the vertices,
that is, an assignment of values $w_i \geq 0$ to the vertices with $\sum_{i=1}^n w_i = 1$.
Our goal is to maximize the function
\[
f(\mathbf{w}) = \sum_{v_i v_j \in E} w_i w_j.
\]

Suppose $\mathbf{w}$ is any distribution, and let $v_i$ and $v_j$ be a pair of nonadjacent
vertices with positive weights $w_i, w_j$. Let $s_i$ be the sum of the weights of all vertices
adjacent to $v_i$, and define $s_j$ similarly for $v_j$, where we may assume that $s_i \geq s_j$.
Now we move the weight from $v_j$ to $v_i$, that is, the new weight of $v_i$ is $w_i + w_j$,
while the weight of $v_j$ drops to 0. For the new distribution $\mathbf{w'}$ we find
\[
f(\mathbf{w'}) = f(\mathbf{w}) + w_j s_i - w_j s_j \geq f(\mathbf{w}).
\]

We repeat this (reducing the number of vertices with a positive weight by one in each step) until
there are no nonadjacent vertices of positive weight anymore. Thus we conclude that there is an
optimal distribution whose nonzero weights are concentrated on a clique, say on a $k$-clique.
Now if, say, $w_1 \geq w_2 > 0$, then choose $w_1' = w_1 - \varepsilon w_1 - w_2$ and
change $w_1$ to $w_1 - \varepsilon$ and $w_2$ to $w_2 + \varepsilon$. The new distribution
$\mathbf{w'}$ satisfies
$f(\mathbf{w'}) = f(\mathbf{w}) + \varepsilon (w_2 s_1 - w_1 s_2) \geq f(\mathbf{w})$, and we infer
that the maximal value of $f(\mathbf{w})$ is attained for $w_i = 1/k$ on a $k$-clique and $w_i = 0$
otherwise. Since a $k$-clique contains $\binom{k}{2}$ edges, we obtain
\[
f(\mathbf{w}) = \binom{k}{2} \frac{1}{k^2} = \frac{1}{2} \left(1 - \frac{1}{k}\right).
\]

Since this expression is increasing in $k$, the best we can do is to set $k = p - 1$
(since $G$ has no $p$-cliques). So we conclude
\[
f(\mathbf{w}) \leq \frac{1}{2} \left(1 - \frac{1}{p - 1}\right)
\]
for any distribution $\mathbf{w}$. In particular, this inequality holds for the \emph{uniform}
distribution given by $w_i = \frac{1}{n}$ for all $i$. Thus we find
\[
\frac{|E|}{n^2} = f\left(\mathbf{w} = \frac{1}{n}\right) \leq \frac{1}{2} \left(1 - \frac{1}{p - 1}\right),
\]
which is precisely (1).
\end{proof}


\begin{theorem}[Fourth Proof]
  \label{ch41proof4}
If a graph $G = (V, E)$ on $n$ vertices has no $p$-clique, $p \geq 2$, then
\[
|E| \leq \left(1 - \frac{1}{p - 1}\right) \frac{n^2}{2}. \tag{1}
\]
\end{theorem}
\begin{proof}
This time we use some concepts from probability theory. Let $G$ be an arbitrary graph on the vertex
set $V = \{v_1, \dots, v_n\}$. Denote the degree of $v_i$ by $d_i$, and write $\omega(G)$ for the
number of vertices in a largest clique, called the clique number of $G$.

\textbf{Claim.} We have $\omega(G) \geq \sum_{i=1}^n \frac{1}{n - d_i}$.

We choose a random permutation $\pi = v_1 v_2 \dots v_n$ of the vertex set $V$, where each
permutation is supposed to appear with the same probability $\frac{1}{n!}$, and then consider the
following set $C_{\pi}$. We put $v_i$ into $C_{\pi}$ if and only if $v_i$ is adjacent to all
$v_j$ $(j < i)$ preceding $v_i$. By definition, $C_{\pi}$ is a clique in $G$. Let $X = |C_{\pi}|$
be the corresponding random variable. We have $X = \sum_{i=1}^n X_i$, where $X_i$ is the indicator
random variable of the vertex $v_i$, that is, $X_i = 1$ or $X_i = 0$ depending on whether
$v_i \in C_{\pi}$ or $v_i \notin C_{\pi}$. Note that $v_i$ belongs to $C_{\pi}$ with respect to the
permutation $v_1 v_2 \dots v_n$ if and only if $v_i$ appears before all $n - 1 - d_i$ vertices which
are not adjacent to $v_i$, or in other words, if $v_i$ is the first among $v_i$ and its $n - 1 - d_i$
non-neighbors. The probability that this happens is $\frac{1}{n - d_i}$,
hence $E X_i = \frac{1}{n - d_i}$.

Thus by linearity of expectation (see ?) we obtain
\[
E(|C_{\pi}|) = E X = \sum_{i=1}^n E X_i = \sum_{i=1}^n \frac{1}{n - d_i}.
\]

Consequently, there must be a clique of at least that size, and this was our claim. To deduce
Turán’s theorem from the claim we use the Cauchy–Schwarz inequality from Chapter \ref{chapter20},
\[
\left( \sum_{i=1}^n a_i b_i \right)^2 \leq \left( \sum_{i=1}^n a_i^2 \right) \left( \sum_{i=1}^n b_i^2 \right).
\]
Set $a_i = \sqrt{n - d_i}$, $b_i = \frac{1}{\sqrt{n - d_i}}$, then $a_i b_i = 1$, and so
\[
n^2 \leq \left( \sum_{i=1}^n (n - d_i) \right) \left( \sum_{i=1}^n \frac{1}{n - d_i} \right) \leq \omega(G) \sum_{i=1}^n (n - d_i). \tag{2}
\]

At this point we apply the hypothesis $\omega(G) \leq p - 1$ of Turán’s theorem.
Using also $\sum_{i=1}^n d_i = 2|E|$ from the chapter on double counting, inequality (2) leads to
\[
n^2 \leq (p - 1)(n^2 - 2|E|),
\]
and this is equivalent to Turán’s inequality.
\end{proof}


\begin{theorem}[Fifth Proof]
  \label{ch41proof5}
If a graph $G = (V, E)$ on $n$ vertices has no $p$-clique, $p \geq 2$, then
\[
|E| \leq \left(1 - \frac{1}{p - 1}\right) \frac{n^2}{2}. \tag{1}
\]
\end{theorem}
\begin{proof}
Let $G$ be a graph on $n$ vertices without a $p$-clique and with a maximal number of edges.

\textbf{Claim.} \textit{G does not contain three vertices $u$, $v$, $w$ such that $vw \in E$, but $uv \notin E$, $uw \notin E$.}

Suppose otherwise, and consider the following cases.

\textbf{Case 1:} $d(u) < d(v)$ or $d(u) < d(w)$.
We may suppose that $d(u) < d(v)$. Then we duplicate $v$, that is, we create a new vertex $v'$
which has exactly the same neighbors as $v$ (but $v'$ is not an edge), delete $u$, and keep the rest unchanged.
The new graph $G'$ has again no $p$-clique, and for the number of edges we find
\[
|E(G')| = |E(G)| + d(v) - d(u) > |E(G)|,
\]
a contradiction.

\textbf{Case 2:} $d(u) \geq d(v)$ and $d(u) \geq d(w)$.
Duplicate $u$ twice and delete $v$ and $w$ (as illustrated in the margin).
Again, the new graph $G'$ has no $p$-clique, and we compute (the $-1$ results from the edge $vw$):
\[
|E(G')| = |E(G)| + 2d(u) - (d(v) + d(w) - 1) > |E(G)|.
\]
So we have a contradiction once more.
A moment’s thought shows that the claim we have proved is equivalent to the statement that
\[
u \sim v : \iff uv \notin E(G)
\]
defines an equivalence relation. Thus $G$ is a complete multipartite graph, $G = K_{n_1, \dots, n_{p-1}}$,
and we are finished.
\end{proof}

\begin{theorem}[Five proofs of Turán's graph theorem]
  \label{turan_graph}
  Collecting the proofs from the chapter...
\end{theorem}
\begin{proof}
  \uses{ch41proof1, ch41proof1, ch41proof1, ch41proof4, ch41proof5}
\end{proof}
